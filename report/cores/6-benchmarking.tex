\section{Benchmarking}
\label{sec:benchmarking}

A quick shortcut for the execution of any solver on all files, describing
an input each, inside a directory can be found in the script ``benchmark.sh''
in the ``scripts/'' directory. Its usage is very simple: each solver is
called with the parameters that should make it behave at its best. Therefore,
we only have to indicate the solver we want to execute, the directory with
all the inputs and where to store the output found for each input. An example
can be found in figure \ref{fig:benchmark-example}.

\begin{figure}[H]
\centering
\begin{minted}{bash}
	./benchmark.sh --solver=CP -i=../inputs/material -o=../outputs/CP
\end{minted}
\centeredcaption{How to use the benchmarking script to execute the solver for
Constraint Programming on the inputs in the ``material'' directory.}
\label{fig:benchmark-example}
\end{figure}

This script will, for every input, invoke the specified solver and check the
quality of the solution found with a \textit{hand-made} solution of that same
instance, that can be found in the directory ``outputs/hand-made''. If the
solution is as good as the \textit{hand-made} then it will output a message
similar to the one in figure \ref{fig:benchmark-verbose:optimal}.

\begin{figure}[H]
\centering
\begin{minted}{bash}
	Executing solver with input file: ../inputs/material/bwp_11_10_1.in
		Optimal solution reached
		In 22.898 seconds
		Current progress:
			Solved optimally    : 85 / 105 ( 80.95% )
			Solved sub-optimally: 15 / 105 ( 14.28% )
			Time elapsed        : 1367.485 seconds
\end{minted}
\centeredcaption{Example of the output of the benchmark script for the
Constraint Programming solver when the solution found is optimal.}
\label{fig:benchmark-verbose:optimal}
\end{figure}

The information shown in figure \ref{fig:benchmark-verbose:optimal} is the following:
it tells whether the optimal solution was reached or not (assuming that the
corresponding output in the ``outputs/hand-made'' directory is indeed optimal),
the time needed to reach that solution, and finally the progress of the script.
In a Pentium IV CPU, 1.8 GHz, with a single core, it took around 23 minutes to
solve a total of 105 inputs, 85 of them optimally and 15 of them suboptimally.
The inputs used to get the message in figure \ref{fig:benchmark-verbose:optimal} were
all the inputs sorted lexicographically from the ``bwp\_3\_3\_1.in'' to the
``bwp\_11\_10\_1.in''. The total output for the same CPU can be found
in the file ``scripts/CP-benchmark-log''\footnote{In a Unix-like environment use
the \textit{cat}, not the \textit{less}, software to display it on the command shell.}.
An example of the message output by the script when the solution is not optimal is
shown in figure \ref{fig:benchmark-verbose:suboptimal}.

\begin{figure}[H]
\centering
\begin{minted}{bash}
	Executing solver with input file: ../inputs/material/bwp_11_8_1.in
		Suboptimal solution:
			Optimal: 12
			CP:      15
		In 27.030 seconds
		Current progress:
			Solved optimally    : 83 / 103 ( 80.58% )
			Solved sub-optimally: 15 / 103 ( 14.56% )
			Time elapsed        : 1334.431 seconds
\end{minted}
\centeredcaption{Example of the output of the benchmark script for the
Constraint Programming solver when the solution found is not optimal.}
\label{fig:benchmark-verbose:suboptimal}
\end{figure}

\hfill

One can design their own inputs and allow the benchmark script to compare
the solution found by any solver to the optimal solution found by hand by
making a file containing the output following the format specified in the
report. Also, one could use the Box Wrapper user interface used to make the
optimal outputs for all the inputs in directory ``inputs/material'' (see
section \ref{sec:box-wrapper}).

\subsection{Constraint Programming}
\label{sec:benchmarking:constraint-programming}

The parameters that we think that makes the Constraint Programming solver to behave
at its best are the following:
\begin{itemize}
	\item Use the mixed heuristic (described in section \ref{sec:constraint-programming:optimum:heuristics}).
	\item The solver will stop as soon as it found one solution.
	\item The solver will stop after 3 seconds of its execution.
	\item The solver will be executed on randomly permuted data $5$ times (needless to
	say that the data will also be permuted $5$ times).
\end{itemize}