\section{Benchmarking}
\label{sec:benchmarking}

A quick shortcut for the execution of any solver on all files inside a directory,
describing an input each, can be found in the script ``benchmark.sh''
in the \textit{scripts/} directory. Its usage is very simple: each solver is
called with the parameters that should make it behave at its best. Therefore,
we only have to indicate the solver we want to execute, the directory with
all the inputs, and where to store the output found for each input. An example
can be found in figure \ref{fig:benchmark-example}.

\begin{figure}[H]
\centering
\begin{lstlisting}[language=bash,basicstyle=\centering]
$ ./benchmark.sh --solver=CP -i=../inputs/material -o=../outputs/CP
\end{lstlisting}
\centeredcaption{How to use the benchmarking script to execute the solver for
Constraint Programming on the inputs in the \textit{inputs/material/} directory.}
\label{fig:benchmark-example}
\end{figure}

All the other solvers available are $CP$ and $LP$. This script will, for every input in the
directory, invoke the specified solver and check the quality of the solution
found with a ``hand-made'' solution of that same instance, that can be found in
the directory \textit{outputs/hand-made/}. If the solution is as good as the
hand-made then it will output a message similar to the one in figure
\ref{fig:benchmark-verbose:optimal}. If it is worse, a message similar to the
one in figure \ref{fig:benchmark-verbose:suboptimal}.

\begin{figure}[H]
\centering
{\scriptsize
\begin{BVerbatim}
Optimal solution reached
In 34.488 seconds
Current progress:
            Solved optimally : 90 / 105 ( 85.71% )
        Solved sub-optimally : 15 / 105 ( 14.28% )
                    Unsolved : 0 / 105 ( 0.00% )
                   Remaining : 105 / 108 ( 97.22% )
          Total time elapsed : 2136.317 seconds
\end{BVerbatim}
}
\centeredcaption{Example of the output of the benchmark script for the
Constraint Programming solver when the solution found is optimal.}
\label{fig:benchmark-verbose:optimal}
\end{figure}

The format of the information shown in figure \ref{fig:benchmark-verbose:optimal}
is as follows: it tells whether the optimal solution was reached or not (assuming
that the corresponding output in the \textit{outputs/hand-made/} directory is indeed
optimal), the time needed to reach that solution, and finally the progress of the
script. In a Pentium IV CPU, 1.8 GHz, with a single core, it took around 36 minutes
to solve a total of 105 inputs, 90 of them optimally and 15 of them suboptimally.
The inputs used to get the message in figure \ref{fig:benchmark-verbose:optimal} were
all the inputs sorted lexicographically from the ``bwp\_3\_3\_1.in'' to the
``bwp\_11\_10\_1.in''. None of them were left unsolved. The total output for the
same CPU can be found in the file ``scripts/CP-p4-1-benchmark-log''\footnote{ In a
Unix-like environment use the command \textit{less -r} to visualise the contents with colour.}.

\begin{figure}[H]
\centering
{\scriptsize
\begin{BVerbatim}
Suboptimal solution:
        Optimal : 12
             CP : 15
    In 45.025 seconds
    Current progress:
            Solved optimally : 88 / 103 ( 85.43% )
        Solved sub-optimally : 15 / 103 ( 14.56% )
                    Unsolved : 0 / 103 ( 14.56% )
                   Remaining : 103 / 108 ( 95.37% )
          Total time elapsed : 2091.806 seconds

\end{BVerbatim}
}
\centeredcaption{Example of the output of the benchmark script for the
Constraint Programming solver when the solution found is not optimal.}
\label{fig:benchmark-verbose:suboptimal}
\end{figure}

One can design their own inputs and allow the benchmark script to compare
the solution found by any solver to the optimal solution found by hand by
making a file containing the optimal output. Obviously, the output must
be in the format specified in the statement. Using the Box Wrapper user
interface could be of help (see section \ref{sec:box-wrapper-gui}).
This simple GUI was used to make the optimal outputs for all the inputs
in directory \textit{inputs/material/}.

\hfill

In the following sections are presented the results for each solver,
on two different CPUs, if possible. The CPUs are an Intel Pentium IV 1.8 GHz
(\textit{p4}), and an Intel i7-6700 HQ 3.5 GHz (\textit{i7}). We also present
results specific for the \textit{i7} using multiple threads. Since the tables
will contain what instances were solved optimally, and which suboptimally, 
it is worth mentioning that this is known for two reasons. One, because the solver
terminated of its own accord, that is, at some point it either decided that
it found the optimal solution (the case for the Linear Programming solver) or
that no better solutions can possibly be found using applying the same algorithm
(the case for the Constraint Programming solver). The other, although the solver
was timeout-terminated, the solution's optimality is known because an optimal
solution is known (the ``hand-made'' solutions). In the case this ``hand-made''
solution did not exist, the optimality of the solver's solution could not be
determined. In this case, the amount of solutions is indicated in \textcolor{red}{red}.

\subsection{Constraint Programming}
\label{sec:benchmarking:constraint-programming}

The parameters used to execute the Constraint Programming solver described in section
\ref{sec:constraint-programming} are the following:
\begin{itemize}
	\item Use the mixed heuristic (described in section \ref{sec:constraint-programming:optimum:heuristics}).
	\item The solver will stop as soon as it finds a solution or 5 seconds of its execution.
	\item The solver will be executed on randomly permuted data $5$ times (needless to
	say that the data will also be permuted $5$ times).
\end{itemize}

With these parameters, the solver has a ``natural'' timeout of \textbf{45} seconds.
That is, each execution of the solver with the data sorted increasingly by area, by
width, decreasingly sorted by area, by width, and each of the 5 random permutations
have a time limit of 5 seconds. With a total of 9 executions that makes
$5 \text{ seconds/execution} \times 9 \text{ executions} = 45 \text{ seconds}$ of time limit.

\hfill

Needless to say that, in spite of having the same time limits, a more powerful computer
is more likely to find better solutions since it can explore the search tree more exhaustively
in the same amount of time. Indeed, this is confirmed in table \ref{table:CP-results}.
The ``(x1)'' and ``(x4)'' refer to the number of threads used for that CPU. However,
the improvement is not as good as one could expect.

\begin{table}[H]
\centering
	\begin{tabular}{rccc}
								& p4			& i7 (x1)	& i7 (x4) \\
		\midrule
		Optimal solutions		& 93 \textcolor{red}{(19)} / 108
											& 97 \textcolor{red}{(13)} / 108
														& 100 \textcolor{red}{(12)} / 108 \\
		Sub-optimal solutions	& 15 / 108	& 11 / 108	&   8 / 108 \\
		Unsolved instances		&  0 / 108	&  0 / 108	&   0 / 108 \\
		Total time				& 37 mins	& 33 mins	& 31 mins \\
	\end{tabular}
	\centeredcaption{Summary presenting the amount of optimally and suboptimally solved
	instances, and unsolved instances over all 108 instances in the \textit{inputs/material/}
	directory. Also, the total execution time is given for the CP solver in two different CPUs.}
	\label{table:CP-results}
\end{table}

The outputs can be found in the \textit{outputs/} directory, inside folders
\textit{CP-p4/}, \textit{CP-i7-1/} and \textit{CP-i7-4/} respectively.
Table \ref{table:CP-results:suboptimalinstances} gives the list of instances
solved suboptimally for each CPU.

\begin{table}[H]
\centering
	\begin{tabular}{ccccccc}
		\multicolumn{3}{c}{Instance name}
					& P4		& i7 (x1)	& i7 (x4)	& Optimal value \\
		\midrule
		 3 &  7 & 1	& 13		&  			&  			& 12 \\
		 3 & 11 & 1	& 31		&  31		& 30			& 29 \\
		 3 & 12 & 1	& 16		&  16		&  			& 15 \\
		 3 & 13 & 1	& 23		&  23		&  			& 22 \\
		 4 & 11 & 1	& 12		&  12		& 12			& 11 \\
		 5 & 12 & 1	& 27		&  27		& 27			& 24 \\
		 6 & 11 & 1	& 15		&  15		&  			& 13 \\
		 6 & 13 & 1	& 36		&  36		& 36			& 35 \\
		 7 & 13 & 1	& 15		&  15		& 15			& 14 \\
		 8 & 13 & 1	& 20		&  20		& 20			& 17 \\
		 9 & 11 & 1	&  7		&   			&  			&  6 \\
		 9 & 13 & 1	& 16		&   			&  			& 15 \\
		10 & 11 & 1	&  6		&   			&  			&  5 \\
		10 & 13 & 1	& 15		&  15		& 15			& 14 \\
		11 &  8 & 1	& 15		&  15		& 14			& 12 \\
	\end{tabular}
	\centeredcaption{Instances in the \textit{inputs/material/} directory
	solved suboptimally and the roll's length found. An empty cell in the
	table indicates that the instance was solved optimally.}
	\label{table:CP-results:suboptimalinstances}
\end{table}

For further details, see the output of the ``benchmark.sh'' script in the files
``scripts/CP-p4-benchmark-log'', ``scripts/CP-i7-1-benchmark-log'' and
``scripts/CP-i7-4-benchmark-log'', respectively.

\subsection{Linear Programming}
\label{sec:benchmarking:linear-programming}

The parameters used to execute the Linear Programming solver are the following:
\begin{itemize}
	\item The solver will stop as soon as it finds the optimal solution or not much
	later than \textbf{45} seconds have passed since the start of its execution.
	\item The solver will use as many available threads as there are in the system.
\end{itemize}

Although the solver was executed on Ubuntu 16.04 its timeout was not applied using
the ``time'' command because the class IloCplex uses an internal clock for it. In
spite of using this timeout, whether executed using 1 or 4 threads, the solver's
execution sometimes took more time: usually around 15 extra seconds, with one single
exception where the solver took around 200 seconds more (lasting up to ~260 seconds).

\hfill

The CPLEX library was obtained through an academical-use only license for a 64-bit system.
For this reason, we can only present results for one CPU only. See table \ref{table:LP-results}
for a summary on the amount of optimally and suboptimally solved instances for \textit{i7}.

\begin{table}[H]
\centering
	\begin{tabular}{rccc}
								& i7 (x1)	& i7 (x4) \\
		\midrule
		Optimal solutions		& 84 \textcolor{red}{(6)} / 108
											& 88 \textcolor{red}{(8)} / 108 \\
		Sub-optimal solutions	& 11 / 108	& 17 / 108 \\
		Unsolved instances		& 13 / 108	& 3 / 108 \\
		Total time				& 34 mins	& 34 mins \\
	\end{tabular}
	\centeredcaption{Summary presenting the amount of optimally and suboptimally solved
	instances, and unsolved instances. Also, the total execution time is given for the LP solver.}
	\label{table:LP-results}
\end{table}

The outputs can be found in the \textit{outputs/} directory, inside folders
\textit{LP-i7-1/} and \textit{LP-i7-4/}. Table \ref{table:LP-results:suboptimalinstances}
gives the list of unsolved instances and instances solved suboptimally.

\begin{table}[H]
\centering
	\begin{tabular}{cccccc}
		\multicolumn{3}{c}{Instance name}
					& i7 (x1)	& i7 (x4)	& Optimal value \\
		\midrule
		 4 & 13 & 1	& 37			& 			& 13 \\
		 5 & 10 & 1	& 16			& 			& 14 \\
		 5 & 13 & 1	& 11			& 			&  6 \\
		 6 & 11 & 1	& 27			& 14			& 13 \\
		 6 & 12 & 1	& 23			& 			&  8 \\
		 6 & 13 & 1	& $x$		& $x$		& 35 \\
		 7 & 12 & 1	& 			& 11			&  5 \\
		 7 & 13 & 1	& 36			& 36			& 14 \\
		 8 &  7 & 1	& 13			& 			&  9 \\
		 8 &  9 & 1	& 25			& 			&  7 \\
		 8 & 10 & 1	& $x$		& 28			&  9 \\
		 8 & 12 & 1	& $x$		& 11			&  7 \\
		 8 & 13 & 1	& $x$		& $x$		& 17 \\
		 9 & 11 & 1	& $x$		& $x$		&  6 \\
		 9 & 12 & 1	& $x$		& 10			&  3 \\
		 9 & 13 & 1	& $x$		& 34			& 15 \\
		10 &  9 & 1	& 22			& 24			&  6 \\
		10 & 10 & 1	& $x$		& 25			&  4 \\
		10 & 11 & 1	& 19			&  9			&  5 \\
		10 & 12 & 1	& $x$		& 16			&  5 \\
		10 & 13 & 1	& 47			& 47			& 14 \\
		11 &  6 & 1	& 			&  5			&  4 \\
		11 &  8 & 1	& $x$		& 32			& 12 \\
		11 & 10 & 1	&  			& 30			&  5 \\
		11 & 11 & 1	& $x$		& 			&  3 \\
		11 & 12 & 1	& $x$		& 28			&  6 \\
		11 & 13 & 1	& $x$		& 24			&  4 \\
	\end{tabular}
	\centeredcaption{Instances in the \textit{inputs/material/} directory
	solved suboptimally and the roll's length found. An empty cell in the
	table indicates that the instance was solved optimally, and an $x$
	that no solution was produced.}
	\label{table:LP-results:suboptimalinstances}
\end{table}

In table \ref{table:LP-results:suboptimalinstances} we can see that using more
computation power helps, but some odd behaviour can be spotted. For example,
instances $(7,12,1)$, $(10,9,1)$, $(11,6,1)$ and $(11,10,1)$  where using four
threads led to a worse solution. However, this is a minor issue since these are
outnumbered by those instances not solved when using one thread but solved,
although with a bad-quality solution, when using four threads.

\hfill

For further details, see the output of the ``benchmark.sh'' script in the files
``scripts/LP-i7-1-benchmark-log'' and ``scripts/LP-i7-4-benchmark-log'', respectively.

