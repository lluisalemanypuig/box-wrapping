\section{Benchmarking}
\label{sec:benchmarking}

A quick shortcut for the execution of any solver on all files, describing
an input each, inside a directory can be found in the script ``benchmark.sh''
in the ``scripts/'' directory. Its usage is very simple: each solver is
called with the parameters that should make it behave at its best. Therefore,
we only have to indicate the solver we want to execute, the directory with
all the inputs and where to store the output found for each input. An example
can be found in figure \ref{fig:benchmark-example}.

\begin{figure}[H]
\centering
\begin{lstlisting}[language=bash,basicstyle=\centering]
$ ./benchmark.sh --solver=CP -i=../inputs/material -o=../outputs/CP
\end{lstlisting}
\centeredcaption{How to use the benchmarking script to execute the solver for
Constraint Programming on the inputs in the \textit{inputs/material} directory.}
\label{fig:benchmark-example}
\end{figure}

This script will, for every input, invoke the specified solver and check the
quality of the solution found with a ``hand-made'' solution of that same
instance, that can be found in the directory \textit{outputs/hand-made/}. If the
solution is as good as the hand-made then it will output a message
similar to the one in figure \ref{fig:benchmark-verbose:optimal}.

\begin{figure}[H]
\centering
\begin{BVerbatim}
Executing solver with input file: ../inputs/material/bwp_11_10_1.in
    Optimal solution reached
    In 34.622 seconds
    Current progress:
        Solved optimally    : 90 / 105 ( 85.71% )
        Solved sub-optimally: 15 / 105 ( 14.28% )
        Total time elapsed  : 2095.598 seconds
\end{BVerbatim}
\centeredcaption{Example of the output of the benchmark script for the
Constraint Programming solver when the solution found is optimal.}
\label{fig:benchmark-verbose:optimal}
\end{figure}

The format of the information shown in figure \ref{fig:benchmark-verbose:optimal}
is as follows: it tells whether the optimal solution was reached or not (assuming
that the corresponding output in the \textit{outputs/hand-made/} directory is indeed
optimal), the time needed to reach that solution, and finally the progress of the
script. In a Pentium IV CPU, 1.8 GHz, with a single core, it took around 23 minutes
to solve a total of 105 inputs, 85 of them optimally and 15 of them suboptimally.
The inputs used to get the message in figure \ref{fig:benchmark-verbose:optimal} were
all the inputs sorted lexicographically from the ``bwp\_3\_3\_1.in'' to the
``bwp\_11\_10\_1.in''. The total output for the same CPU can be found
in the file ``scripts/CP-p4-benchmark-log''\footnote{ In a Unix-like environment use
the command \textit{less -r} to visualise the contents with colour.}.
An example of the message output by the script when the solution is not optimal is
shown in figure \ref{fig:benchmark-verbose:suboptimal}.

\begin{figure}[H]
\centering
\begin{BVerbatim}
Executing solver with input file: ../inputs/material/bwp_11_8_1.in
    Suboptimal solution:
        Optimal: 12
        CP:      15
    In 45.022 seconds
    Current progress:
        Solved optimally    : 88 / 103 ( 85.43% )
        Solved sub-optimally: 15 / 103 ( 14.56% )
        Total time elapsed  : 2050.956 seconds
\end{BVerbatim}
\centeredcaption{Example of the output of the benchmark script for the
Constraint Programming solver when the solution found is not optimal.}
\label{fig:benchmark-verbose:suboptimal}
\end{figure}

\hfill

One can design their own inputs and allow the benchmark script to compare
the solution found by any solver to the optimal solution found by hand by
making a file containing the output following the format specified in the
report. Also, one could use the Box Wrapper user interface used to make the
optimal outputs for all the inputs in directory \textit{inputs/material/}
(see section \ref{sec:box-wrapper-gui}).

\subsection{Constraint Programming}
\label{sec:benchmarking:constraint-programming}

The parameters used to execute the Constraint Programming solver described in section
\ref{sec:constraint-programming} are the following:
\begin{itemize}
	\item Use the mixed heuristic (described in section \ref{sec:constraint-programming:optimum:heuristics}).
	\item The solver will stop as soon as it found one solution.
	\item The solver will stop after 5 seconds of its execution.
	\item The solver will be executed on randomly permuted data $5$ times (needless to
	say that the data will also be permuted $5$ times).
\end{itemize}

With these parameters, the solver has a ``natural'' timeout of \textbf{45} seconds.
That is, each execution of the solver with the data sorted increasingly by area, by
width, decreasingly sorted by area, by width, and each of the 5 random permutations
have a time limit of 5 seconds. With a total of 9 executions that makes $5 \cdot 9 = 45$
seconds of time limit.

\subsubsection{Results}
\label{sec:benchmarking:constraint-programming:results}

Needless to say that depending on the machine this software is executed on,
in spite of having the same time limits, a more powerful computer is more likely
to find better solutions since it can explore the search tree more exhaustively
in the same amount of time. Indeed, this is confirmed in table \ref{table:CP-results}.

\begin{table}[H]
\centering
	\begin{tabular}{rcc}
								& Pentium IV 1.8 GHz & i7-6700 HQ 3.5 GHz \\
		\midrule
		Optimal solutions		& 93 / 108			 & 97 / 108 \\
		Sub-optimal solutions	& 15 / 108			 & 11 / 108 \\
		Total time				& 37 mins			 & 33 mins \\
	\end{tabular}
	\centeredcaption{Comparison between two different CPUs over all the 108 instances
	in the \textit{inputs/material/} directory.}
	\label{table:CP-results}
\end{table}

The outputs can be found in the \textit{outputs/} directory, inside folders
\textit{CP-p4/} and \textit{CP-i7/} respectively. Table \ref{table:CP-results:suboptimalinstances}
gives the list of instances solved suboptimally for each CPU.

\begin{table}[H]
\centering
	\begin{tabular}{cccccc}
		\multicolumn{3}{c}{Instance name}
					& Pentium IV 1.8 GHz	& i7-6700 HQ 3.5 GHz	& Optimal value \\
		\midrule
		3 & 7 & 1	& 13					& 						& 12 \\
		3 & 11 & 1	& 31					& 31					& 29 \\
		3 & 12 & 1	& 16					& 16					& 15 \\
		3 & 13 & 1	& 23					& 23					& 22 \\
		4 & 11 & 1	& 12					& 12					& 11 \\
		5 & 12 & 1	& 27					& 27					& 24 \\
		6 & 11 & 1	& 15					& 15					& 13 \\
		6 & 13 & 1	& 36					& 36					& 35 \\
		7 & 13 & 1	& 15					& 15					& 14 \\
		8 & 13 & 1	& 20					& 20					& 17 \\
		9 & 11 & 1	& 7						&						& 6 \\
		9 & 13 & 1	& 16					&						& 15 \\
		10 & 11 & 1	& 6						&						& 5 \\
		10 & 13 & 1	& 15					& 15					& 14 \\
		11 & 8 & 1	& 15					& 15					& 12 \\
	\end{tabular}
	\centeredcaption{Instances in the \textit{inputs/material/} directory
	solved suboptimally. An empty cell in the table indicates that the instance
	was solved optimally.}
	\label{table:CP-results:suboptimalinstances}
\end{table}
