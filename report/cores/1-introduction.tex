\section{Introduction}
\label{sec:introduction}

The Box Wrapping Problem (BWP) is easy to formulate: given a list of $N \in \Natural$ boxes
$b_i$ each of dimensions $(w_i, l_i)$, where $w_i \in \Natural$ and $l_i \in \Natural$
are the width and length of box $b_i$ respectively, and a roll width $W \in \Natural$,
find the coordinates of  the top-left corner of each box $(\Xtl[i], \Ytl[i])$ so that
the length of roll $L \in \Natural$ is minimised. We call the set of top-left corners, i.e.,
the solution, the placement of all the boxes. In this project we consider only the
2-dimensional case, that is, boxes actually have eight sides that have to be wrapped but,
to make it simple, we consider that a box is just a rectangle. In addition to this, boxes
can be rotated.

\hfill

An example of an instance for this problem is the following: given the list of boxes each
of dimensions $(1,1)$, $(1,1)$, $(2,1)$, $(1,3)$, and $W=3$, find the placement of these
boxes that minimises the roll length. The optimal solution is clearly the one shown
in figure \ref{fig:example-placement}.

\begin{figure}[H]
	\centering
	\psscalebox{0.75 0.75}
    {\begin{pspicture}(0,-2.4125)(3.595,2.4125)
\definecolor{colour0}{rgb}{1.0,0.37254903,0.0}
\psframe[linecolor=black, linewidth=0.01, fillstyle=solid,fillcolor=green, dimen=outer](3.305,0.6925)(2.505,-0.9075)
\psframe[linecolor=black, linewidth=0.01, fillstyle=solid,fillcolor=yellow, dimen=outer](1.705,0.6925)(0.905,-0.1075)
\psframe[linecolor=black, linewidth=0.01, fillstyle=solid,fillcolor=colour0, dimen=outer](2.505,0.6925)(1.705,-0.1075)
\psframe[linecolor=black, linewidth=0.01, fillstyle=solid,fillcolor=red, dimen=outer](3.305,1.4925)(0.905,0.6925)
\psline[linecolor=black, linewidth=0.04](0.905,1.4925)(0.905,-2.1075)
\psline[linecolor=black, linewidth=0.04](0.905,1.4925)(3.305,1.4925)
\psline[linecolor=black, linewidth=0.04](1.705,1.4925)(1.705,-2.1075)
\psline[linecolor=black, linewidth=0.04](2.505,1.4925)(2.505,-2.1075)
\psline[linecolor=black, linewidth=0.04](3.305,1.4925)(3.305,-2.1075)
\psline[linecolor=black, linewidth=0.04](0.905,0.6925)(3.305,0.6925)
\psline[linecolor=black, linewidth=0.04](0.905,-0.1075)(3.305,-0.1075)
\psline[linecolor=black, linewidth=0.04](0.905,-0.9075)(3.305,-0.9075)
\psdots[linecolor=black, dotsize=0.06](2.105,-1.3075)
\psdots[linecolor=black, dotsize=0.06](2.105,-1.7075)
\psdots[linecolor=black, dotsize=0.06](2.905,-1.3075)
\psdots[linecolor=black, dotsize=0.06](2.905,-1.7075)
\psdots[linecolor=black, dotsize=0.06](1.305,-1.3075)
\psdots[linecolor=black, dotsize=0.06](1.305,-1.7075)
\psdots[linecolor=black, dotsize=0.06](1.305,-2.1075)
\psdots[linecolor=black, dotsize=0.06](2.105,-2.1075)
\psdots[linecolor=black, dotsize=0.06](2.905,-2.1075)
\psline[linecolor=black, linewidth=0.04, arrowsize=0.05291667cm 2.0,arrowlength=1.4,arrowinset=0.0]{->}(0.905,1.8925)(3.305,1.8925)
\psline[linecolor=black, linewidth=0.04](0.505,1.4925)(0.505,-0.1075)
\rput(2.105,2.2925){$W$}
\rput(0.105,0.2925){$L$}
\psdots[linecolor=black, dotsize=0.06](0.505,-0.5075)
\psdots[linecolor=black, dotsize=0.06](0.505,-0.9075)
\psdots[linecolor=black, dotsize=0.06](0.505,-1.3075)
\psline[linecolor=black, linewidth=0.04, arrowsize=0.05291667cm 2.0,arrowlength=1.4,arrowinset=0.0]{->}(0.505,-1.7075)(0.505,-2.1075)
\rput(3.505,2.0925){$x$}
\rput(0.305,-2.3075){$y$}
\end{pspicture}
}
    %\includegraphics[scale=•]{•}=0.175]{example-placement}
	\centeredcaption{Optimal placement for $W=3$ and boxes $(1,1)$, $(1,1)$, $(2,1)$,
	$(1,3)$.}
	\label{fig:example-placement}
\end{figure}

There are many ways to solve this problem, but in this project we will solve it from
three different points of view: constraint programming (see section \ref{sec:constraint-programming}),
linear programming (see section\ref{sec:linear-programming}), and satisfiability (see
section \ref{sec:satisfiability}). Since all these approaches require a mathematical modelling
of the problem (variables related by constraints), we give, in section \ref{sec:modelling},
a generic model that should describe the problem well enough to solve it with each technology.
However, each technology may require some extra variables or constraints.