\section{Linear Programming}
\label{sec:linear-programming}

In this part of the project the problem is solved using the Linear Programming
paradigm. In particular, we used the library CPLEX for C++ (version 12.7)
(see \cite{CplexWEB}) in order to implement the mathematical model that models
the problem and helps us solve it without having to implement our own algorithm.

\hfill

The use of this paradigm has its advantages and disadvantages.
One of the disadvantages is that it is not as expressive as the constraint
programming paradigm. For example, it does not easily allow logical expressions,
since they are not linear expressions. For example, in order to implement the
logical or ($a \vee b$) we have to introduce some new variables to help the
solver satisfy one of the two conditions.

\lluis{Give example for OR expression}
\lluis{Give example for AND expression}
\lluis{How to implement an implication?}

\lluis{Advantages: it allows easy optimisation}

\hfill

In spite of this, we do not to reformulate the constraints epxlained in
section \ref{sec:modelling} because CPLEX does the necessary transformations
seamlessly.

\lluis{After writing how are the constraints implemented, say that:
the constraint (3) in the simple solver is not needed, although without
it the solver assigns invalid values to the variables that model the cells
of each box.

\hfill

Look at commit\\
https://github.com/lluisalemanypuig/box-wrapping/commit/9a2c91a2fd8344e3e078649da105cd4d7227da71
}

