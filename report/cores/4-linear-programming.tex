\section{Linear Programming}
\label{sec:linear-programming}

In this part of the project the problem is solved using the Linear Programming
paradigm. In particular, we used the library CPLEX for C++ (version 12.7)
(see \cite{CplexWEB}) in order to implement the mathematical model that models
the problem and helps us solve it without having to implement our own algorithm.

\hfill

The use of this paradigm has its advantages and disadvantages.
One of the the advantages is that we can naturally model optimisation problems
with their own objective function (whose value has to be maximised or minimised).
One of the disadvantages is that it is not as expressive as the constraint
programming paradigm. For example, it does not easily allow logical expressions,
since they are not linear. For example, in order to implement the
logical or ($a \vee b$) we have to introduce some new variables to help the
solver satisfy one of the two conditions. Now follows an explanation on how to
transform several logical expressions into linear expressions, assuming them
to be integer-valued for all feasible solution, using simple transformations
that relate each expression to 0/1 variables. Let $\delta_1$, $\delta_2$ be such
variables:

\begin{itemize}	
	\item $(\delta_1 = 1) \vee (\delta_2 = 1)$. Impose $\delta_1 + \delta_2 \ge 1$.
	
	\item $(\delta_1 = 0) \vee (\delta_2 = 1)$.	Impose $\delta_1 \le \delta_2$.
	
	\item $(\delta_1 = 0) \vee (\delta_2 = 0)$.	Impose $\delta_1 + \delta_2 \le 1$.
	
	\item $(\delta_1 = 1) \wedge (\delta_2 = 1)$. Impose $\delta_1 + \delta_2 = 2$.
	
	\item $(\delta_1 = 0) \wedge (\delta_2 = 1)$. Impose $1 - \delta_1 + \delta_2 = 2$.	
	
	\item $(\delta_1 = 0) \wedge (\delta_2 = 0)$. Impose $\delta_1 + \delta_2 = 0$.
	
	\item $(\delta_1 = b_1) \Longrightarrow (\delta_2 = b_1) \equiv (\delta_1 = 1 - b_1) \vee (\delta_2 = b_2)$,
	for any two values $b_1,b_2 \in \mathbb{B}$.
	
	\item $(\delta_1 = b_1) \Longleftrightarrow (\delta_2 = b_2)$. Impose the two logical constraints
	$(\delta_1 = b_1) \Longrightarrow (\delta_2 = b_2)$ and $(\delta_2 = b_2) \Longrightarrow (\delta_1 = b_1)$,
	for any two values $b_1,b_2 \in \mathbb{B}$.
	
\end{itemize}

Relating linear expressions through logical operators is slightly more complicated.
If $a_1^Tx \le b_1$ and $a_2^Tx \le b_2$ are two integer-valued linear expressions
for all feasible solution $x$, to model the logical expression:
\[
(a_1^Tx \le b_1) \;\Box\; (a_2^Tx \le b_2)
\]
where $\Box$ is any binary logical operator $\vee,\wedge,\Longrightarrow,\Longleftrightarrow$,
we have to apply the following procedure: for each of the two linear expressions introduce a 0/1
variable. Let $\delta_1$ and $\delta_2$ be these variables defined so that:
\[
(\delta_1 = 1) \Longleftrightarrow (a_1^Tx \le b_1), \qquad 
(\delta_2 = 1) \Longleftrightarrow (a_2^Tx \le b_2)
\]
It only remains to impose the logical constraint $(\delta_1 = 1) \;\Box\; (\delta_2 = 1)$,
and use linear expressions for the two double implications:
\[
a_i^Tx - b_i \le \mathcal{U}_i(1 - \delta_i), \qquad a_i^Tx - b_i \ge (\mathcal{L}_i - 1)\delta_i + 1
\]
where $\mathcal{U}_i$ and $\mathcal{L}_i$ are upper and lower bounds, respectively, for
the expression $a_i^Tx - b_i$, for $i\in\{1,2\}$.

\hfill

In spite of logical constraints not being linear expressions, we do not to reformulate
the constraints explained in section \ref{sec:modelling} because CPLEX does the necessary
transformations automatically.

\subsection{Implementation}
\label{sec:linear-programming:implementation}

In order to implement the variables detailed in section \ref{sec:modelling:variables},
three arrays of integer variables were used, each of them containing the variables described
in items \ref{var:box-cell}, \ref{var:box-corner} and \ref{var:box-rotated}. Taking
$L$ as the upper bound on the roll's length calculated with equation (\ref{eq:upper-bound-L}),
the arrays are initialised as follows:

\begin{enumerate}
	\item Array for variables in \ref{var:box-cell}:
    
	{\NOINDENT \begin{lstlisting}
	box_cell = IloNumVarArray(env, N*W*L, 0, 1, ILOINT);
	\end{lstlisting}}
    
	\item Array for variables in \ref{var:box-corner}:
    
	{\NOINDENT \begin{lstlisting}
	box_corner = IloNumVarArray(env, N*W*L, 0, 1, ILOINT);
	\end{lstlisting}}
    
	\item Array for variables in \ref{var:box-rotated}:
    
	{\NOINDENT \begin{lstlisting}
	box_rotated = IloNumVarArray(env, N, 0, 1, ILOINT);
	\end{lstlisting}}
    
\end{enumerate}

Following the CPLEX's requirements for a proper use of its solvers, we required the
use of these three objects:

{\NOINDENT \begin{lstlisting}
IloEnv env;
IloModel model(env);
IloCplex cplex(model);
\end{lstlisting}}

Assuming the implementation of the following functions to facilitate
access to the arrays of variables:

{\NOINDENT \begin{lstlisting}
inline IloNumVar X(size_t b, size_t x, size_t y) const { return box_corner[b*W*L + y*W + x]; }
inline IloNumVar C(size_t b, size_t x, size_t y) const { return box_cell[b*W*L + y*W + x]; }
inline IloNumVar R(size_t b) const { return box_rotated[b]; }

inline IloNumVar X(size_t b, size_t x, size_t y) { return box_corner[b*W*L + y*W + x]; }
inline IloNumVar C(size_t b, size_t x, size_t y) { return box_cell[b*W*L + y*W + x]; }
inline IloNumVar R(size_t b) { return box_rotated[b]; }
\end{lstlisting}}

the implementation of the constraints detailed in section \ref{sec:modelling:constraints}
using the CPLEX library is as follows:

\begin{enumerate}
	\item All boxes must be in the solution (constraint formalised in
	\ref{constr:box-placed}, equation \ref{eq:constraint:all-boxes-used}),
	also read as ``assign a top-left corner to each box''.
    
	{\NOINDENT \begin{lstlisting}
	for (int b = 0; b < N; ++b) {
		IloExpr exa_X(env);
		for (length i = 0; i < L; ++i) {
			for (width j = 0; j < W; ++j) {
				exa_X += X(b,i,j);
			}
		}
		model.add(exa_X == 1);
		exa_X.end();
	}
	\end{lstlisting}}
    
	\item Boxes cannot overlap (constraint formalised in
	\ref{constr:no-overlap}, constraint \ref{eq:constraint:one-box-cell}).
    
	{\NOINDENT \begin{lstlisting}
	for (length i = 0; i < L; ++i) {
		for (width j = 0; j < W; ++j) {
			IloExpr amo_C(env);
			for (int b = 0; b < N; ++b) {
				amo_C += C(b,i,j);
			}
			model.add(amo_C <= 1);
			amo_C.end();
		}
	}
	\end{lstlisting}}
    
	\item Depending on their rotation, boxes occupy certain cells of the roll
	(constraint formalised in \ref{constr:box-rot-span}). For those cells $(x,y)$ within
	bounds for $b$-th box:

	\begin{itemize}
		\item For square boxes (constraint \ref{eq:span-cells:square-boxes}):
		{\NOINDENT \begin{lstlisting}
		for (length i = y; i <= y + b_length - 1; ++i) {
			for (width j = x; j <= x + b_width - 1; ++j) {
				model.add(
					IloIfThen(env, (X(b,x,y) == 1), (C(b,j,i) == 1))
				);
			}
		}
		model.add(R(b) == 0);
		\end{lstlisting}}
		
		\item For rectangular boxes (constraints \ref{eq:span-cells:rectangular-boxes:0} and
		\ref{eq:span-cells:rectangular-boxes:1}):
		{\NOINDENT \begin{lstlisting}
		for (length i = y; i <= y + b_length - 1; ++i) {
			for (width j = x; j <= x + b_width - 1; ++j) {
				model.add(
					IloIfThen(env, ((R(b) == 0) && (X(b,x,y) == 1)), (C(b,j,i) == 1))
				);
			}
		}
		\end{lstlisting}}
		{\NOINDENT \begin{lstlisting}
		for (length i = y; i <= y + b_width - 1; ++i) {
			for (width j = x; j <= x + b_length - 1; ++j) {
				model.add(
					IloIfThen(env, ((R(b) == 1) && (X(b,x,y) == 1)), (C(b,j,i) == 1))
				);
			}
		}
		\end{lstlisting}}
	\end{itemize}
    
	\item Depending on their rotation, boxes cannot occupy certain cells of the
	roll (constraint formalised in \ref{constr:box-forbid}). For those cells $(x,y)$
	out of bounds for $b$-th box:
    
	\begin{itemize}
		\item For square boxes (constraint \ref{eq:forbid-span-cells:square-boxes}):
		{\NOINDENT \begin{lstlisting}
		model.add(X(b,x,y) == 0);
		\end{lstlisting}}
		
		\item For rectangular boxes (constraints \ref{eq:forbid-span-cells:rectangular-boxes:0}
		and \ref{eq:forbid-span-cells:rectangular-boxes:1}):
		{\NOINDENT \begin{lstlisting}
		model.add(
			IloIfThen(env, (R(b) == 0), (X(b,x,y) == 0))
		);
		\end{lstlisting}}
		{\NOINDENT \begin{lstlisting}
		model.add(
			IloIfThen(env, (R(b) == 1), (X(b,x,y) == 0))
		);
		\end{lstlisting}}
	\end{itemize}
    
\end{enumerate}

All the ``linear'' expressions above are the only required according to the
general modelling of the problem described in section \ref{sec:modelling}. However,
the linear programming paradigm requires an extra constraint: each box can not
occupy more cells than its area. This constraint is required for the following
reason: since is logically true that ``false implies true'' we can have many
cells $c$ of the roll assigned to some box $b$ even if its top-left corner is not
assigned to any cell $t$ nearby\footnote{By ``nearby'' we mean ``placed at a cell so that
according to the box's rotation and its dimensions cell $c$ falls within the
area that should be occupied by $b$ when its top-left corner is assigned to $t$.''}.
This way we may encounter an assignation that makes a box have, not only more cells
than it should, but also cells assigned throughout the completely scattered.
See figure \ref{fig:invalid-roll} for an illustration of this phenomenon.

\begin{figure}[H]
\centering
\begin{BVerbatim}
   | W              | W              | W              | W              | W       
 L |  0  1  2     L |  0  1  2     L |  0  1  2     L |  0  1  2     L |  0  1  2
-------------    -------------    -------------    -------------    -------------
 0 | 13 13 12    11 | 11 13  8    22 | 13 13 13    33 | 12  2  1    44 |  7  7 12
 1 |  6  6 12    12 | 13 13 13    23 |  1 12 13    34 | 13  2  1    45 |  7 13 10
 2 |  6  6 12    13 | 13  5  5    24 |  4  8 13    35 | 11  2 13
 3 |  6  6 12    14 | 13  5  5    25 | 11  4 13    36 | 12 13 11
 4 |  6  6 12    15 | 12  5  5    26 | 13  4 13    37 | 12 13 11
 5 |  8  8 13    16 | 13  5  5    27 | 13 11 13    38 |  3 13 11
 6 |  8  8 13    17 | 13 12 13    28 | 13 12  4    39 |  3 11 11
 7 |  8  8 11    18 |  7 12 12    29 |  4  4 13    40 |  3 13 11
 8 |  8  8 13    19 | 13 13 11    30 |  4  4  4    41 |  7  7  7
 9 |  8  8 13    20 | 13 13 13    31 |  4  4  4    42 |  7  7 12
10 | 12 13  9    21 |  4 13 13    32 |  4  4  1    43 |  7  7 13
\end{BVerbatim}
\centeredcaption{Invalid contents of the roll without rotations and optimisation for instance
\textit{instances/material/bwp\_3\_13\_1.in}}
\label{fig:invalid-roll}
\end{figure}

In spite of this, this only affects to the extraction of the solution and yet it 
is a minor problem: the construction of the solution only requires knowing the
rotation and the position of the top-left corner of each box. Knowing these two,
and having access to the input data, we can easily construct the valid roll\footnote{The
actual solution to the problem, as specified in the statement, is not the roll
itself but only the top-left and bottom-right corners of each box.}
(see figure \ref{fig:valid-roll}). Therefore, the constraint is not only superfluous,
but also its implementation and addition to the model only slows the performance down.
However, we still need to implement the variables and constraints to avoid boxes
overlapping.s

\begin{figure}[H]
\centering
\begin{BVerbatim}
   | W             | W             | W             | W             | W          
 L |  0 1  2     L |  0 1  2     L |  0 1  2     L |  0 1  2     L |  0 1  2    
------------    ------------    ------------    ------------    ------------    
 0 |  . 6  6    11 |  7 7  9    22 |  . .  .    33 |  . . 11    44 | 12 .  .
 1 |  . 6  6    12 |  7 7  .    23 |  . .  .    34 |  . . 11    45 |  . .  .
 2 |  . 6  6    13 |  7 7  .    24 |  . . 10    35 |  1 . 11
 3 |  . 6  6    14 |  7 7  .    25 |  . . 10    36 |  1 . 11
 4 |  . .  2    15 |  . .  .    26 |  . . 10    37 |  1 .  .
 5 |  . .  2    16 |  . 3  3    27 |  4 4 10    38 |  . .  .
 6 |  . .  2    17 |  . 3  3    28 |  4 4 10    39 |  . .  .
 7 |  . .  .    18 |  . 3  3    29 |  4 4  .    40 | 12 5  5
 8 |  . .  .    19 |  . 3  3    30 |  4 4  .    41 | 12 5  5
 9 |  . 8  .    20 |  . .  .    31 |  . .  .    42 | 12 5  5
10 |  . .  .    21 |  . .  .    32 |  . . 11    43 | 12 5  5
\end{BVerbatim}
\centeredcaption{Valid contents of the roll without rotations and optimisation for instance
\textit{instances/material/bwp\_3\_13\_1.in}}
\label{fig:valid-roll}
\end{figure}

\subsection{Finding the optimium solution}
\label{sec:linear-programming:optimum}

\lluis{
Explain the new variables introduced and the objective function
}


