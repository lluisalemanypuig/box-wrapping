\section{Constraint Programming}
\label{sec:constraint-programming}

In this part of the project, this problem is solved using the Constraint Programming
paradigm. In particular, we used the library Gecode for C++ (version 6.0.0)
(see \cite{GecodeWEB}) in order to implement the mathematical model that would model
the problem and help us solve it without implement our own algorithm.

\hfill

For this, we used 3 arrays of Boolean variables, each of them containing the
variables described in items \ref{var:box-cell}, \ref{var:box-corner} and
\ref{var:box-rotated} in section \ref{sec:modelling:variables}. Taking $L$
as the upper bound on the roll's length calculated with equation
(\ref{eq:upper-bound-L}), the arrays are initialised as follows:

\begin{enumerate}
    \item Array for variables \ref{var:box-cell}:
    
    \begin{lstlisting}
    box_cell = BoolVarArray(*this, N*W*L, 0, 1);
    \end{lstlisting}
    
    \item Array for variables \ref{var:box-corner}:
    
    \begin{lstlisting}
    box_corner = BoolVarArray(*this, N*W*L, 0, 1);
    \end{lstlisting}
    
    \item Array for variables \ref{var:box-rotated}:
    
    \begin{lstlisting}
    box_rotated = BoolVarArray(*this, N, 0, 1);
    \end{lstlisting}
    
\end{enumerate}

Also, the implementation of the constraints is as follows:

\begin{enumerate}
    \item All boxes must be in the solution (constraint formalised in
    \ref{constr:box-placed}).
    
    \lluis{Include Gecode code}
    
    \item Boxes cannot overlap (constraint formalised in \ref{constr:no-overlap}).
    
    \lluis{Include Gecode code}
    
    \item Depending on their rotation, boxes occupy certain cells of the roll
    (constraint formalised in \ref{constr:box-rot-span}).
    
    \lluis{Include Gecode code}
    
    \item Depending on their rotation, boxes cannot occupy certain cells of the
    roll (constraint formalised in \ref{constr:box-forbid}).
    
    \lluis{Include Gecode code}
\end{enumerate}