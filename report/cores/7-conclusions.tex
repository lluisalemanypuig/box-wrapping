\section{Conclusions}
\label{sec:conclusions}

The aim of this project was to use three different strategies to solve the Box Wrapping
Problem, an optimisation problem with a huge combinatorial space. Two of the
strategies applied are not defined to solve optimisation problems (the Constraint
Programming paradigm, in section \ref{sec:constraint-programming}, and the
Satisfiability paradigm, in section \ref{sec:satisfiability}) and the other
was defined for this particular purpose (the Linear Programming paradigm, in
section \ref{sec:linear-programming}). The latter, in spite of being specifically
made to solve such type of problems, lacks the ability to solve problems with
a significant constraint-solving component, one of the reasons we had to apply,
not the ``pure'' Linear Programming approach, but the Mixed-Integer Linear Programming.

\hfill

This project has shown that all of these strategies can be applied to solve this problem,
though some of them more easily than others: the Constraint Programming technique only
required the addition of extra constraints (see section \ref{sec:constraint-programming:optimum}),
the Linear Programming technique the definition of an objective function (see
\ref{sec:linear-programming:optimum}) and Satisfiability required the implementation of a
whole framework that uses a SAT solver as a black box to find an optimal solution (see section
\ref{sec:satisfiability:optimum}). The difficulty in finding the optimal solution increases
with every technique.

\hfill

Despite of this, the most difficult technique seems to be most efficient.
Table \ref{table:benchmark:SAT-results} shows that with the Satisfiability
paradigm we could find the optimal solution to 99 instances (the other 9 are known
to be optimal due to the existence of a hand-made solution to the corresponding
instance, see discussion at the beginning of section \ref{sec:benchmarking}), while tables
\ref{table:benchmark:CP-results} and \ref{table:LP-results} show, respectively, that
the Constraint Programming approach could find 88 optimal instances (and 12 presumably
optimal) and that the Linear Programming approach could find 80 optimal instances (and
8 presumably optimal). Therefore, it seems that Satisfiability seems to be the best
approach to tackle the BWP.

\hfill

However, it is worth mentioning that the license of \textbf{CPLEX} (see \cite{CplexWEB}),
used for the Linear Programming part, is ``academical'': it is quite likely that a complete
license of this software might be more efficient and even outperform the \textbf{Gecode}
library (see \cite{GecodeWEB}) and/or \textbf{lingeling} (see \cite{lingeling}). Moreover,
the use of a more powerful CPU seems, in some cases, to be completely useless since with it
we can only obtain the same set of solutions, only slightly faster. Aiming at solving all
instances might require improving significantly the modelling of the model.

\hfill

It could be argued that the mathematical modelling of the problem (see section
\ref{sec:modelling}) is rather simple. For example, constraints to stop the solvers from
trying to improve a soluion by swaping two identical boxes (avoidance of symmetries) should
have been implemented. Admittedly, this could have speeded up the solvers, but it has been
proven that a few and simple heuristics can also help, even a great deal, in finding optimal
solutions. Furthermore, the lack of these constraints does not seem to be an issue in the
Satisfiability part. However, these solvers were only tested on rather small instances.

\hfill

Taking everything into account, it seems that the Satisfiability approach seems to be the
best for solving this problem, as long as there is no more evidence in favour of the
Linear Programming approach.




