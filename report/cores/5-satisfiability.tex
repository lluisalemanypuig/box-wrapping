\section{Satisfiability}
\label{sec:satisfiability}

This part of the project is aimed at solving the BWP using the
satisfiability paradigm, that is, construct a Boolean formula
in Conjunctive Normal Form (CNF) using a number of Boolean
variables and solve it. The solution is a truth assignment to
the variables. The way this solution is obtained is by means
of a SAT solver. In this project, lingeling \cite{lingeling},
developed by Biere, is the solver that will be used.

\hfill

This time, this paradigm has mostly only disadvantages: not only
we cannot find an optimal solution directly just by finding
a truth assignment to the variables, but also the ``language''
is less expressive, or, rather, more rigid since it only allows
Boolean clauses in CNF than the one used in Constraint or Linear
Programming. However, the constraints and variables that have
to be implemented are exactly the same as the ones formalised
in section \ref{sec:modelling} and, as will be explained in the
coming sections, optimality can be reached using fairly simple
methods.

\hfill

By less expressive we mean the following: take, for example,
the constraint \ref{constr:box-placed} formalised in section
\ref{sec:modelling:constraints}. This was implemented in
sections \ref{sec:constraint-programming} and \ref{sec:linear-programming}
following equation \ref{eq:constraint:all-boxes-used} literally.
This time we can not do that. This constraint formalised the idea
that each box has to have its top-left corner assigned to exactly
one cell. In Boolean satisfiability this can be seen as ``given $n$
Boolean variables, exactly one of the $n$ variables $x_1,\cdots,x_n$ has
to be true''.



This type of constraint
can be interpreted as ``of the $n$ variables $x_1,\cdots,x_n$ exactly
one of them''. Implementing this
constraint requires two types of constraints: ``at least one''
and ``at most one''. If we manage to implement them using two
Boolean expressions in CNF each, the conjunction of the two
will give use the ``exactly one'' constraint. The former is
extremely easy: given $n$ variables $x_1,\cdots,x_n$, the
``at least one'' constraint ($x_1+\cdots+x_n\le 1$) is obtained
from the disjunction of all of the positive literals $l_1,\cdots,l_n$.

\subsection{Implementation}

