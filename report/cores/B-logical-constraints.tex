\section{Examples logical constraints}
\label{sec:examples-logical-constraints}

In this section are shown two simple examples on how to transform logical
constraints that relate Boolean variables into linear expressions.

\subsection{3 or}
\label{sec:examples-logical-constraints:3-ors}

Assume three Boolean variables $a,b,c$ related as follows:
\begin{equation}
\label{eq:B:3-ors}
(a = 1) \vee (b = 0) \vee (c = 1)
\end{equation}
and we want to express the same condition but with a series of
linear constraints. First, use the associative property of the logical $\vee$
to obtain: $((a = 1) \vee (b = 0)) \vee (c = 1)$.
Now, introduce a new Boolean variable $\delta$ so that
\[
(\delta = 1) \Longleftrightarrow ((a = 1) \vee (b = 0))
\;\equiv\;
(\delta = 1) \Longleftrightarrow (b \le a)
\]
Now, using the transformations explained in section \ref{sec:linear-programming}
we obtain two new linear expressions
\[
b - a \le \mathcal{U}(1 - \delta), \quad b - a \ge (\mathcal{L} - 1)\delta + 1
\]
Taking $\mathcal{U}=1$ and $\mathcal{L}=-1$ as upper and lower bounds for $b - a$
we get
\begin{equation}
\label{eq:B:expressions-b-lt-a-delta}
b - a \le 1 - \delta, \quad b - a \ge -2\delta + 1
\end{equation}
Our claim is that if the three linear expressions in \ref{eq:B:three-lin-expr}
are satisfied then the logical constraint in \ref{eq:B:3-ors} will be true.
\begin{equation}
\label{eq:B:three-lin-expr}
\delta + c \ge 1,\qquad b - a \le 1 - \delta,\qquad b - a \ge -2\delta + 1
\end{equation}

For this, we can test all 8 possible combinations of values for $a,b,c$. For each pair
of values for $a,b$ the algorithm that solves the linear constraints gives a value to
variable $\delta$ that satisfies the expressions in \ref{eq:B:expressions-b-lt-a-delta}.
This value is then used to check that the constraint $\delta + c \ge 1$ is satisfied
for the given value of $c$. If it is, then the logical constraint in \ref{eq:B:3-ors}
is satisfied, that is, it is true that either $a=1$, $b=0$, $c=1$.

\begin{table}[H]
\centering
	\begin{tabular}{cccccccc}
	  & a & b & c & $(b - a \le 1 - \delta) \wedge (b - a \ge -2\delta + 1)$                     & $\delta$ & $\delta + c \ge 1$ \\
	\midrule
	* & 0 & 0 & 0 & $(0 \le 1 - \delta) \wedge (0 \ge -2\delta + 1) \equiv 1/2 \le \delta \le 1$ &        1 & $\bullet$ \\
	* & 0 & 0 & 1 &                                                                              &        1 & $\bullet$ \\
	  & 0 & 1 & 0 & $(1 \le 1 - \delta) \wedge (1 \ge -2\delta + 1) \equiv   0 \le \delta \le 0$ &        0 &           \\
	* & 0 & 1 & 1 &                                                                              &        0 & $\bullet$ \\
	* & 1 & 0 & 0 & $(-1 \le 1 - \delta) \wedge (-1 \ge -2\delta + 1) \equiv 1 \le \delta \le 2$ &        1 & $\bullet$ \\
	* & 1 & 0 & 1 &                                                                              &        1 & $\bullet$ \\
	* & 1 & 1 & 0 & $(0 \le 1 - \delta) \wedge (0 \ge -2\delta + 1) \equiv 1/2 \le \delta \le 1$ &        1 & $\bullet$ \\
	* & 1 & 1 & 1 &                                                                              &        1 & $\bullet$ \\
	\end{tabular}
\centeredcaption{The cases where the constraint \ref{eq:B:3-ors} is satisfied (marked with ``*'')
and the cases where the constraints in \ref{eq:B:three-lin-expr} are satisfied (marked with $\bullet$).}
\label{table:B:3-linear-expressions}
\end{table}

\subsection{Several and}
\label{sec:examples-logical-constraints:several-ands}

Transforming a series of logical ``and'' ($\wedge$) operations is fairly trivial.
Given Boolean variables $x_1,\cdots,x_n$ each equal to some Boolean value $b_1,\cdots,b_n$,
the logical constraint
\begin{equation}
\label{eq:B:several-ands}
\bigwedge_{i=1}^{n} (x_i = b_i)
\end{equation}
is transformed into the following equality:
\begin{eqnarray}
\label{eq:B:several-ands:linear}
\sum_{i : b_i=1} x_i + \sum_{i : b_i=0} 1 - x_i = n
\end{eqnarray}
which in turn is transformed into two inequalities $\ge$,$\le$.

\hfill

The only assignment to the variables $x_1,\cdots,x_n$ that satisfies equation
\ref{eq:B:several-ands:linear} is $b_1,\cdots,b_n$. The proof is done by contradiction.
Assume there are $p$ variables that must be $1$ and $z$ variables that must be $0$.
If the assignment is $b_1,\cdots,b_n$ then:
\[
\sum_{i : b_i=1} x_i + \sum_{i : b_i=0} 1 - x_i = p + z = n
\]
If $b_i=1$ then let $x_i=0$. In this case we have that:
\[
\sum_{i : b_i=1} x_i + \sum_{i : b_i=0} 1 - x_i = p - 1 + z \neq n
\]
Conversely, if $b_i=0$ let $x_i=1$. Then, we have
\[
\sum_{i : b_i=1} x_i + \sum_{i : b_i=0} 1 - x_i = p + z - 1 \neq n
\]
